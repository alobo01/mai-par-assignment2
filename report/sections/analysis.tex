\section{Analysis of the problem}

We studied several problems, each presenting increasing levels of difficulty. In this section, we provide a detailed description of 
all of them. 

The robot is assumed to hold only one object at a time, whether it is an ingredient, a tool, or a prepared dish. Its storage capacity is
 limited, which constrains its ability to carry objects. Despite this limitation, we assume that the robot can perform various actions,
  such as assembling, cooking, cutting, or mixing, while holding an object, as it can be placed inside the robot without restricting its
   arms, which allows it to perform the aforementioned actions.

Subsequently, we will introduce the common predicates and actions used across all three scenarios considered in this project.
\subsection{Basic problem}
Predicates used:
\begin{itemize}
  \item $ (\text{robot-at } ?r \text{ - robot } ?loc \text{ - location}) $: Represents the location of a specific robot 'r' in an area of the kitchen 'loc'.
  \item $ (\text{ingredient-at } ?\text{ingredient - ingredient } ?\text{loc - location}) $: This predicate signifies the presence of an 'ingredient' in an area 'loc'.
  \item $ (\text{tool-at } ?\text{tool - tool } ?\text{loc - location}) $: This predicate indicates that the instrument 'tool' is in the area of the kitchen 'loc'.
  \item $ (\text{ingredient-prepared } ?\text{ingredient - ingredient}) $: This predicate indicates that the ingredient 'ingredient' has been successfully prepared.
  \item $ (\text{dish-assembled } ?\text{dish - dish}) $: Indicates that the dish 'dish' has been assembled from its ingredients.
  \item $ (\text{dish-plated } ?\text{dish - dish } ?\text{loc - location}) $: Denotes that the dish 'dish' has been plated and placed at location 'loc'.
  \item $ (\text{tool-clean } ?\text{tool - tool}) $: This predicate denotes that the tool 'tool' is clean.
  \item $ (\text{holding-ingredient } ?\text{r - robot } ?\text{ingredient - ingredient}) $: Indicates that the robot 'r' is holding the ingredient 'ingredient'.
  \item $ (\text{holding-dish } ?\text{r - robot } ?\text{dish - dish}) $: Indicates that the robot 'r' is holding the assembled dish 'dish'.
  \item $ (\text{holding-tool } ?\text{r - robot } ?\text{tool - tool}) $: Indicates that the robot 'r' is holding the tool 'tool'.
  \item $ (\text{adjacent } ?\text{loc1 - location } ?\text{loc2 - location}) $: Indicates that the kitchen areas 'loc1' and 'loc2' are adjacent.
  \item $ (\text{used-in } ?\text{ingredient - ingredient } ?\text{dish - dish}) $: This predicate represents if the ingredient 'ingredient' has been used to prepare the dish 'dish'.
  \item $ (\text{need-mix } ?\text{ingredient - ingredient}) $: Denotes that the ingredient 'ingredient' requires mixing.
  \item $ (\text{need-cook } ?\text{ingredient - ingredient}) $: Denotes that the ingredient 'ingredient' requires cooking.
  \item $ (\text{need-cut } ?\text{ingredient - ingredient}) $: Denotes that the ingredient 'ingredient' requires cutting.
\end{itemize}
Actions used:
\begin{itemize}
    \item \textbf{pick-up-ingredient:} The robot picks up an ingredient at a specified location.
    \item \textbf{pick-up-tool:} The robot picks up a tool at a specified location.
    \item \textbf{move:} The robot moves from one location to an adjacent location.
    \item \textbf{drop-ingredient:} The robot drops the ingredient it is holding at a specified location.
    \item \textbf{drop-tool:} The robot drops the tool it is holding at a specified location.
    \item \textbf{mix:} The robot mixes an ingredient (e.i. rice) at the mixing area (MIXA), provided the ingredient needs mixing and is not already prepared.
    \item \textbf{cook:} The robot cooks an ingredient (e.i. rice) at the cooking area (CA), provided it has been mixed and needs cooking.
    \item \textbf{cut:} The robot cuts an ingredient at the cutting area (CTA) using a clean tool, provided the ingredient needs cutting.
    \item \textbf{clean-tool:} The robot cleans a tool in the dishwashing area (DWA).
    \item \textbf{assemble-dish:} The robot assembles a dish using prepared ingredients at a specified location.
    \item \textbf{carrying-dish:} The robot carries an assembled dish, provided it is not already holding any other items.
    \item \textbf{plate-dish:} The robot plates an assembled dish at the serving area (SVA).
\end{itemize}

