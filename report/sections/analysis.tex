\section{Analysis of the problem}

\subsection{Basic problem}
Predicates used:
\begin{itemize}
  \item $ (\text{robot-at } ?r \text{ - robot } ?loc \text{ - location}) $: Representes the localitation of a specific robot 'r', in an area of the kitchen 'loc'.
  \item $ (\text{ingredient-at } ?\text{ingredient - ingredient } ?\text{loc - location} $): This predicate signifies the presence of an 'ingridient' in an area 'loc'.
  \item $ (\text{tool-at } ?\text{tool - tool } ?\text{loc - location} $): This predicate indicates that the instrument 'tool' is in the area of the kitchen 'loc'.
  \item $ (\text{ingredient-prepared } ?\text{ingredient - ingredient} $): This predicate indicates that the finished item of food 'dish' has been successfully assembled.
  \item $ (\text{dish-assembled } ?\text{dish - dish} $): ??
  \item $ (\text{dish-plated } ?\text{dish - dish } ?\text{loc - location} $): It denotes that the instrument 'tool' is clean.
  \item $ (\text{tool-clean } ?\text{tool - tool} $): This predicaate denotes that the robot 'r' is holding the ingridient 'ingridient'.
  \item $ (\text{holding } ?\text{r - robot } ?\text{item - item} $): This predicate represents that the robot 'r' is holding 'item', it can be a tool or an ingredient.
  \item $ (\text{adjacent } ?\text{loc1 - location } ?\text{loc2 - location} $): Comprobation if the areas of the kitchen 'loc1' and 'loc2' are adjacent.
  \item $ (\text{used-in } ?\text{ingredient - ingredient } ?\text{dish - dish} $): This predicates represents if the item 'ingrdient' has been used to prepare the recipe 'dish'.
\end{itemize}