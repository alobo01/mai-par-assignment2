\documentclass{article}
\usepackage[a4paper, left=2.5cm, right=2cm, top=3cm, bottom=3cm]{geometry}
\usepackage{graphicx}
\usepackage{cite}
\usepackage{amssymb}
\usepackage{amsmath}
\usepackage{ulem}
\usepackage{subcaption}
\usepackage{caption}
\usepackage{float}
\usepackage{hyperref}
\usepackage{placeins}


% Title
\title{Planning and Aproximate Reasoning: Robot Chef Task}
\author{
	María del Carmen Ramírez, Pedro Agúndez and Antonio Lobo.
}
\date{November 10, 2024}

\begin{document}
	
	\maketitle

\section{Introduction}

Planning an effective work schedule in sectors like catering or the restaurant industry, where numerous tasks must be carried out quickly and efficiently, can be essential to achieving final objectives optimally.

In this work, we will focus on the design and implementation of a planner in PDDL (Planning Domain Definition Language) to modelate how a robot performs tasks equivalent to those carried out by waiters and chefs in an Asian food restaurant. Specifically, a total of three problems will be presented along with their respective domains, and an analysis will be conducted on how optimal solutions are obtained in each scenario in order to meet the objectives set for each of the three problems. 
\section{Analysis of the problem}

We studied several problems, each presenting increasing levels of difficulty. In this section, we provide a detailed description of 
all of them. 

The robot is assumed to hold only one object at a time, whether it is an ingredient, a tool, or a prepared dish. Its storage capacity is
 limited, which constrains its ability to carry objects. Despite this limitation, we assume that the robot can perform various actions,
  such as assembling, cooking, cutting, or mixing, while holding an object, as it can be placed inside the robot without restricting its
   arms, which allows it to perform the aforementioned actions.

Subsequently, we will introduce the common predicates and actions used across all three scenarios considered in this project.
\subsection{Basic problem}
Predicates used:
\begin{itemize}
  \item $ (\text{robot-at } ?r \text{ - robot } ?loc \text{ - location}) $: Represents the location of a specific robot 'r' in an area of the kitchen 'loc'.
  \item $ (\text{ingredient-at } ?\text{ingredient - ingredient } ?\text{loc - location}) $: This predicate signifies the presence of an 'ingredient' in an area 'loc'.
  \item $ (\text{tool-at } ?\text{tool - tool } ?\text{loc - location}) $: This predicate indicates that the instrument 'tool' is in the area of the kitchen 'loc'.
  \item $ (\text{ingredient-prepared } ?\text{ingredient - ingredient}) $: This predicate indicates that the ingredient 'ingredient' has been successfully prepared.
  \item $ (\text{dish-assembled } ?\text{dish - dish}) $: Indicates that the dish 'dish' has been assembled from its ingredients.
  \item $ (\text{dish-plated } ?\text{dish - dish } ?\text{loc - location}) $: Denotes that the dish 'dish' has been plated and placed at location 'loc'.
  \item $ (\text{tool-clean } ?\text{tool - tool}) $: This predicate denotes that the tool 'tool' is clean.
  \item $ (\text{holding-ingredient } ?\text{r - robot } ?\text{ingredient - ingredient}) $: Indicates that the robot 'r' is holding the ingredient 'ingredient'.
  \item $ (\text{holding-dish } ?\text{r - robot } ?\text{dish - dish}) $: Indicates that the robot 'r' is holding the assembled dish 'dish'.
  \item $ (\text{holding-tool } ?\text{r - robot } ?\text{tool - tool}) $: Indicates that the robot 'r' is holding the tool 'tool'.
  \item $ (\text{adjacent } ?\text{loc1 - location } ?\text{loc2 - location}) $: Indicates that the kitchen areas 'loc1' and 'loc2' are adjacent.
  \item $ (\text{used-in } ?\text{ingredient - ingredient } ?\text{dish - dish}) $: This predicate represents if the ingredient 'ingredient' has been used to prepare the dish 'dish'.
  \item $ (\text{need-mix } ?\text{ingredient - ingredient}) $: Denotes that the ingredient 'ingredient' requires mixing.
  \item $ (\text{need-cook } ?\text{ingredient - ingredient}) $: Denotes that the ingredient 'ingredient' requires cooking.
  \item $ (\text{need-cut } ?\text{ingredient - ingredient}) $: Denotes that the ingredient 'ingredient' requires cutting.
\end{itemize}
Actions used:
\begin{itemize}
    \item \textbf{pick-up-ingredient:} The robot picks up an ingredient at a specified location.
    \item \textbf{pick-up-tool:} The robot picks up a tool at a specified location.
    \item \textbf{move:} The robot moves from one location to an adjacent location.
    \item \textbf{drop-ingredient:} The robot drops the ingredient it is holding at a specified location.
    \item \textbf{drop-tool:} The robot drops the tool it is holding at a specified location.
    \item \textbf{mix:} The robot mixes an ingredient (e.i. rice) at the mixing area (MIXA), provided the ingredient needs mixing and is not already prepared.
    \item \textbf{cook:} The robot cooks an ingredient (e.i. rice) at the cooking area (CA), provided it has been mixed and needs cooking.
    \item \textbf{cut:} The robot cuts an ingredient at the cutting area (CTA) using a clean tool, provided the ingredient needs cutting.
    \item \textbf{clean-tool:} The robot cleans a tool in the dishwashing area (DWA).
    \item \textbf{assemble-dish:} The robot assembles a dish using prepared ingredients at a specified location.
    \item \textbf{carrying-dish:} The robot carries an assembled dish, provided it is not already holding any other items.
    \item \textbf{plate-dish:} The robot plates an assembled dish at the serving area (SVA).
\end{itemize}


\section{Results}
In all the scenarios described in the previous section, the results were obtained using a PDDL planner implemented in Julia. 
Specifically, the A* search algorithm was implemented. The planner initializes the problem state using the initstate function and 
defines the goals using MinStepsGoal, optimizing for minimal steps to achieve the objectives, while the planning process employs an A$*$ 
planner (AStarPlanner) with the HAddR heuristic to guide the search.


The implementation of Julia was mandatory due to...


The final path obtained for each case can be seen in ... 
\section{Discusion}
In this section, it is analyzed the plans obtained in each scenario and how they were achieved. In order to provide a insight into the 
efficiency and complexity of the searching process, several key performance parameters were analyzed, as it is shown in Table \ref{tab:tabla}. 

\begin{table}[H]
    \centering
    \begin{tabular}{|l|c|c|c|}
    \hline
    Problem         & Basic & Substitution & Priorization order \\ \hline
    Plan length     &  41     &    42          &        18            \\ \hline
    Expanded states &    33413   &     34710         &  2784                  \\ \hline
    Search time (s)    & 36.0221934      &    58.560908       &    0.9147218           \\ \hline
    Total time (s)     &   36.0221938     &     58.5609156           & 0.9147222 \\ \hline
    \end{tabular}\label{tab:tabla}
    \caption{Performance of the planners across different test cases.}
\end{table}



\section{Conclusion}
This work addresses three different problems and draws two key conclusions, aligning with theoretical insights. Specifically, while domain-specific models are more costly to develop, they provide significant gains in efficiency. Our approach evolved across three stages:

\begin{enumerate}
	\item \textbf{General Domain:} Initially, we pursued a general domain capable of handling various processes (such as cutting, cleaning, and assembling) by simply adjusting the problem configuration. However, this approach proved inefficient, as each action involved numerous parameters, greatly increasing processing time.
	
	\item \textbf{Specific Domain:} Next, we shifted to a domain-specific model, creating distinct actions for each process. Although this approach required more development effort, it led to substantial efficiency improvements, aligning with the theory that tailored domains can optimize performance.
	
	\item \textbf{Implementing Constants:} We observed that the planner frequently evaluated multiple locations for the same action, despite fixed locations being required for certain processes. By introducing constants and removing the location parameter, we achieved further efficiency gains, reinforcing the benefits of domain-specific constraints.
\end{enumerate}

This progression underscores that, while domain-specific models demand a higher development investment, they offer crucial efficiency advantages, consistent with theoretical expectations.

This study also reveals that creating a separate plan for each dish and then prioritizing the serving order through a secondary planner is far more efficient. By breaking down the task into individual subplans, the planner avoids navigating an overwhelming number of states to achieve its goal, which significantly enhances processing speed. Our findings show that planning for large, complex tasks as a single sequence is inefficient; instead, dividing the task into manageable subtasks allows for simpler and more effective planning.

Additionally, we experimented with the Backward A$*$ algorithm; however, the results were less efficient than the approach presented here. This further emphasizes the advantages of the domain-specific adjustments and task subdivision strategies we employed.

\end{document}